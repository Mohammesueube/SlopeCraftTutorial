\documentclass[UTF8]{ctexart}
\usepackage{geometry}
\usepackage{fancyhdr}
\usepackage{verbatim}
\usepackage{enumerate}

\geometry{papersize={210mm,297mm}}
\geometry{left=2.5cm,right=2.5cm,top=2.5cm,bottom=2.5cm}

\setlength{\headheight}{13pt}

\title{SlopeCraft v3.6教程}
\author{TokiNoBug}
\date{\today}

\begin{comment}
\pagestyle{fancy}

\lhead{\author}
\chead{\title}
\rhead{\date}

\lfoot{}
\cfoot{\thepage}
\rfoot{}

\renewcommand{\headrulewidth}{0.4pt}
\renewcommand{\headwidth}{\textwidth}
\renewcommand{\footrulewidth}{0pt}

\end{comment}

\begin{document}
    \maketitle
    SlopeCraft v3.6是一个重大更新,整个程序都被重构了一遍,增加了许多硬核的功能,解决了不少棘手的麻烦:v3.6实现了期盼已久的有损压缩功能,可以把地图画总高度压缩到几乎任何值,彻底解决了地图画超出限高的问题;v3.6加入了“搭桥”功能,可以在每个水平面内,用玻璃方块连接所有方块,形成通路,辅助玩家建造……
    
    除此以外,v3.6还修改了界面的操作逻辑。加之上一版教程还是v3.1,我认为有必要写个新的教程。

    这篇教程不是“傻瓜式”的,我只介绍必须介绍的,强调必须强调的。

    \pagebreak
    \section{SlopeCraft v3.6更新内容}
    新内容:

    \textbf{SlopeCraftL3.dll}——SlopeCraft的内核开发成了通过接口类调用的\textbf{动态链接库},随应用程序一同发布,包含了立体、平板、纯文件地图画和墙面像素画的核心功能,理论上应该能被C++、Java、Python等语言调用(顶多再套一层接口),后来者可以\textbf{少造轮子}了!

    新功能:
    \begin{enumerate}
        \item 智能有损压缩——\textbf{彻底解决超限高问题}
        \item 搭桥——\textbf{方便建造立体地图画}
        \item 防火/防末影人——用玻璃块包围\textbf{易燃方块}和\textbf{可被小黑偷走}的方块
        \item 导出为.NBT格式——\textbf{Litematica}和\textbf{原版结构方块}可以用,WorldEdit不知
        \item 自定义方块列表——更加自由,想要什么方块\textbf{自己加}
        \item 检查更新、反馈Bug——请认准GitHub仓库
        \item 更加完善的报错系统——\textbf{报错信息不是天书!}
        \item 启动时自动设置语言——默认简体中文,如果系统语言没有汉语,那就英文
        \item settings.json——启动时的配置文件
        \item 新的像素画类型:墙面像素画——不伦不类的怪物,偏偏很多人想要这个功能
    \end{enumerate}
    修复Bug:
    \begin{enumerate}
        \item 修复了无法禁用部分颜色的Bug——重写了方块列表的逻辑 
    \end{enumerate}
    优化和其他更改:
    \begin{enumerate}        
        \item 优化了转化图片和搭桥的性能——感谢AbrasiveBoar902
        \item 删除了可去掉的“确认”按钮,节省操作
        \item 等待用户操作时进度条不再显示为繁忙状态,更符合操作习惯
    \end{enumerate}
\end{document}